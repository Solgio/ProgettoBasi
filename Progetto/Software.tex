\section{Applicazione Software}
\begin{comment}
\begin{lstlisting}[caption={Example C Code}]
... definizioni e inclusioni necessarie ...

const char *STATUS[] = {
    "Consegnato",
    "In Consegna",
    "Spedito",
    "Autorizzato",
    "Fase di Controllo"};

... altri enum per evitare errori di battitura ...

... funzioni per la connessione al database ...

void printResult(int rows, int cols, PGresult *res)
{
    for (int i = 0; i < rows; i++)
    {
        for (int j = 0; j < cols; j++)
        {
            printf("%s\t\t", PQgetvalue(res, i, j));
        }
        printf("\n");
    }
}

int main(int argc, char **argv)
{
    .
    .
    .
    int end = 0;
    while (end != 1)
    {
        printf("Premi invio per continuare...\n");
        getchar();

        int scelta;
        printf("Che query vuoi eseguire?\n A tua disposizione abbiamo queste 5 opzioni:\n");
        printf("\t1. Media del costo dei pacchi di un corriere di un certo grado che sono passati per una filiale di un certo tipo\n\t2. Numero pacchi per bundle con un determinato tipo di assicurazione\n\t3. Corrieri con tempi di consegna maggiori alla media\n\t4. Corrieri con bundle con pacchi assicurati o regalo e costo specifici\n\t5. Status attuale dei pacchi\n");
        printf("Qual'è la tua scelta? (1-5): ");
        scanf("%d", &scelta);
        while (scelta < 1 || scelta > 5)
        {
            printf("\tHai scelto... molto MALE!\n\tTrovo insopportabile la sua mancanza di Input corretto\n");
            printf("Le scelte sono 5, riprova: ");
            scanf("%d", &scelta);
        }
        printf("\tHai scelto molto bene!  (cit)\n\n");
        PGresult *res;

        if (scelta == 1)
        {
            const char *query1 = "SELECT corriere, AVG(p.costo) AS costo_medio "
                                 "FROM pacco AS p "
                                 "JOIN tracking AS t ON t.id_pacco = p.id "
                                 "JOIN filiale AS f ON t.città_check_point = f.città AND t.nome_check_point=f.nome "
                                 "JOIN corriere AS c ON p.corriere = c.cf "
                                 "WHERE f.tipo = $1 AND c.grado = $2 "
                                 "GROUP BY corriere;";
            res = PQprepare(conn, "query1", query1, 2, NULL);
            if (PQresultStatus(res) != PGRES_COMMAND_OK)
            {
                fprintf(stderr, "Prepare failed: %s", PQerrorMessage(conn));
                PQclear(res);
            }
            PQclear(res);

            int grado;
            int tipo_filiale;
            printf("Scegli il Tipo di Filiale. Le opzioni sono:\n");
            for (int i = 0; i < 3; i++)
            {
                printf("\t%d. %s\n", i + 1, TIPO_FILIALE[i]);
            }
            printf("Quale tipo vuoi? (1-3): ");
            scanf("%d", &tipo_filiale);
            while (tipo_filiale < 1 || tipo_filiale > 3)
            {
                printf("\tHai scelto... molto MALE!\n\tTrovo insopportabile la sua mancanza di Input corretto\n");
                printf("Quale grado vuoi? (1-4): ");
                scanf("%d", &tipo_filiale);
            }

            printf("Scegli il Grado del Corriere. Le opzioni sono:\n");
            for (int i = 0; i < 4; i++)
            {
                printf("\t%d. %s\n", i + 1, GRADO_CORRIERE[i]);
            }
            printf("Quale grado vuoi? (1-4): ");
            scanf("%d", &grado);
            while (grado < 1 || grado > 4)
            {
                printf("\tHai scelto... molto MALE!\n\tTrovo insopportabile la sua mancanza di Input corretto\n");
                printf("Quale grado vuoi? (1-4): ");
                scanf("%d", &grado);
            }

            const char *paramValues[2] = {
                TIPO_FILIALE[tipo_filiale - 1],
                GRADO_CORRIERE[grado - 1]};

            res = PQexecPrepared(conn, "query1", 2, paramValues, NULL, NULL, 0);
            checkResults(res, conn);

            // Stampa i risultati
            int rows = PQntuples(res);
            int cols = PQnfields(res);

            printf("\n\t\tMedia del costo dei pacchi di un corriere %s che sono passati per una filiale %s:\n", GRADO_CORRIERE[grado - 1], TIPO_FILIALE[tipo_filiale - 1]);
            printf("ID Corriere\t\t\tCosto Medio\n");
            printf("-----------\t\t\t-----------\n");
            printResult(rows, cols, res);
            PQclear(res);
        }

        if (scelta == 2)
        {
            const char *query2 = "SELECT b.id_bundle, COUNT(*) as n_garanzia_tre_anni "
                                 "FROM bundle AS b "
                                 "JOIN pacco AS p ON b.id_contenuto = p.id "
                                 "WHERE p.tipo_assicurazione = $1 "
                                 "GROUP BY b.id_bundle";
            res = PQprepare(conn, "query2", query2, 1, NULL);

            int assicurazione;
            printf("Scegli il tipo di assicurazione. Le opzioni sono:\n");
            for (int i = 0; i < 3; i++)
            {
                printf("\t%d. %s\n", i + 1, TIPO_ASSICURAZIONE[i]);
            }
            printf("Quale tipo vuoi? (1-3): ");
            scanf("%d", &assicurazione);
            while (assicurazione < 1 || assicurazione > 3)
            {
                printf("\tHai scelto... molto MALE!\n\tTrovo insopportabile la sua mancanza di Input corretto\n");
                printf("Quale grado vuoi? (1-3): ");
                scanf("%d", &assicurazione);
            }

            const char *paramValue = TIPO_ASSICURAZIONE[assicurazione - 1];
            res = PQexecPrepared(conn, "query2", 1, &paramValue, NULL, NULL, 0);
            checkResults(res, conn);

            // Stampa i risultati
            int rows = PQntuples(res);
            int cols = PQnfields(res);

            printf("\nNumero pacchi per bundle con %s:\n", TIPO_ASSICURAZIONE[assicurazione - 1]);
            printf("ID Bundle\t\t\tNumero pacchi Assicurati\n");
            printf("---------\t\t\t------------------------\n");
            printResult(rows, cols, res);
            PQclear(res);
        }

        if (scelta == 3)
        {
            const char *query3 = "SELECT p.corriere, AVG(h.differenza) AS media_corriere "
                                 "FROM ( "
                                 "    SELECT t1.id_pacco, p1.corriere, MAX(t1.data_ora) - p1.dataora_ordine AS differenza "
                                 "    FROM "
                                 "        tracking AS t1 "
                                 "    JOIN pacco AS p1 ON p1.id = t1.id_pacco "
                                 "    WHERE t1.status = 'Consegnato' "
                                 "    GROUP BY t1.id_pacco, p1.dataora_ordine, p1.corriere "
                                 ") AS h "
                                 "JOIN pacco AS p ON p.id = h.id_pacco "
                                 "GROUP BY p.corriere "
                                 "HAVING AVG(h.differenza) > ( "
                                 "    SELECT AVG(differenza) "
                                 "    FROM ( "
                                 "        SELECT t1.id_pacco, MAX(t1.data_ora) - p2.dataora_ordine AS differenza "
                                 "        FROM tracking AS t1 "
                                 "        JOIN pacco AS p2 ON p2.id = t1.id_pacco "
                                 "        WHERE t1.status = 'Consegnato' "
                                 "        GROUP BY t1.id_pacco, p2.dataora_ordine "
                                 "    ) AS m "
                                 ") "
                                 "ORDER BY media_corriere DESC;";
            PGresult *res = PQexec(conn, query3);
            checkResults(res, conn);

            // Stampa i risultati
            int rows = PQntuples(res);
            int cols = PQnfields(res);

            printf("\nCorrieri più lenti del tempo medio di consegna:\n");
            printf("ID Corriere\t\t\tTempo di consegna medio\n");
            printf("-----------\t\t\t-----------------------\n");

            printResult(rows, cols, res);
            PQclear(res);
        }

        if (scelta == 4)
        {
            int costo;
            printf("Scegli il costo per il filtraggio dei risultati: ");
            scanf("%d", &costo);
            char costo_str[12];
            snprintf(costo_str, sizeof(costo_str), "%d", costo);

            int filtro;
            printf("Vuoi filtrare per pacchi assicurati o regalo?    (0=assicurati, 1=regalo): ");
            scanf("%d", &filtro);
            while (filtro < 0 || filtro > 1)
            {
                printf("\n Input non valido, riprova: ");
                printf("Vuoi filtrare per pacchi assicurati o regalo?    (0=assicurati, 1=regalo): ");
                scanf("%d", &filtro);
            }
            if (filtro == 0)
            {
                const char *query4 = "SELECT cf, count(*) "
                                     "FROM corriere as c "
                                     "JOIN pacco as p ON p.costo>$1 AND p.corriere=cf "
                                     "JOIN ( "
                                     "      SELECT b.id_bundle, COUNT(*) as n_garanzia_tre_anni "
                                     "      FROM bundle AS b "
                                     "      JOIN pacco AS p ON b.id_contenuto = p.id "
                                     "      WHERE p.tipo_assicurazione = $2"
                                     "      GROUP BY b.id_bundle "
                                     ") as b ON p.id=b.id_bundle "
                                     "GROUP BY cf";
                res = PQprepare(conn, "query4", query4, 2, NULL);
                int assicurazione;

                printf("Scegli il tipo di assicurazione. Le opzioni sono:\n");
                for (int i = 0; i < 3; i++)
                {
                    printf("\t%d. %s\n", i + 1, TIPO_ASSICURAZIONE[i]);
                }
                printf("Quale tipo vuoi? (1-3): ");
                scanf("%d", &assicurazione);
                while (assicurazione < 1 || assicurazione > 3)
                {
                    printf("\tHai scelto... molto MALE!\n\tTrovo insopportabile la sua mancanza di Input corretto\n");
                    printf("Quale grado vuoi? (1-3): ");
                    scanf("%d", &assicurazione);
                }

                const char *paramValues[2] = {
                    costo_str,
                    TIPO_ASSICURAZIONE[assicurazione - 1]};
                res = PQexecPrepared(conn, "query4", 2, paramValues, NULL, NULL, 0);
            }
            else
            {
                int caso;
                printf("Cosa vuoi dalla query:\n\t1. Solo tipo carta\n\t2. Solo dedica\n\t3. Entrambi\n Cosa preferisci? (1-3): ");
                scanf("%d", &caso);
                while (caso < 1 || caso > 3)
                {
                    printf("\n Input non valido, riprova: ");
                    printf("Cosa vuoi dalla query:\n\t1. Solo tipo carta\n\t2. Solo dedica\n\t3. Entrambi\n Cosa preferisci? (1-3): ");
                    scanf("%d", &filtro);
                }

                if (caso == 1)
                {
                    const char *query4a = "SELECT cf, count(*) "
                                          "FROM corriere as c "
                                          "JOIN pacco as p ON p.costo>$1 AND p.corriere=cf "
                                          "JOIN ( "
                                          "      SELECT b.id_bundle, COUNT(*) as n_garanzia_tre_anni "
                                          "      FROM bundle AS b "
                                          "      JOIN pacco AS p ON b.id_contenuto = p.id "
                                          "      WHERE p.tipo_carta = $2"
                                          "      GROUP BY b.id_bundle "
                                          ") as b ON p.id=b.id_bundle "
                                          "GROUP BY cf";
                    res = PQprepare(conn, "query4a", query4a, 2, NULL);
                    int assicurazione;

                    int carta;
                    printf("Scegli il tipo di carta. Le opzioni sono:\n");
                    for (int i = 0; i < 3; i++)
                    {
                        printf("\t%d. %s\n", i + 1, TIPO_CARTA[i]);
                    }
                    printf("Quale tipo vuoi? (1-3): ");
                    scanf("%d", &carta);
                    while (carta < 0 || carta > 3)
                    {
                        printf("\tHai scelto... molto MALE!\n\tTrovo insopportabile la sua mancanza di Input corretto\n");
                        printf("Quale carta vuoi? (1-3): ");
                        scanf("%d", &carta);
                    }

                    const char *paramValues[2] = {
                        costo_str,
                        TIPO_CARTA[carta - 1]};
                    res = PQexecPrepared(conn, "query4a", 2, paramValues, NULL, NULL, 0);
                }

                else if (caso == 2)
                {
                    const char *query4b = "SELECT cf, count(*) "
                                          "FROM corriere as c "
                                          "JOIN pacco as p ON p.costo>$1 AND p.corriere=cf "
                                          "JOIN ( "
                                          "      SELECT b.id_bundle, COUNT(*) as n_garanzia_tre_anni "
                                          "      FROM bundle AS b "
                                          "      JOIN pacco AS p ON b.id_contenuto = p.id "
                                          "      WHERE p.dedica = $2"
                                          "      GROUP BY b.id_bundle "
                                          ") as b ON p.id=b.id_bundle "
                                          "GROUP BY cf";
                    res = PQprepare(conn, "query4b", query4b, 2, NULL);

                    char dedica[100];
                    printf("Inserisci la dedica: ");
                    scanf("%99s", dedica);

                    const char *paramValues[2] = {
                        costo_str,
                        dedica};
                    res = PQexecPrepared(conn, "query4b", 2, paramValues, NULL, NULL, 0);
                }

                else
                {
                    const char *query4c = "SELECT cf, count(*) "
                                          "FROM corriere as c "
                                          "JOIN pacco as p ON p.costo>$1 AND p.corriere=cf "
                                          "JOIN ( "
                                          "      SELECT b.id_bundle, COUNT(*) as n_garanzia_tre_anni "
                                          "      FROM bundle AS b "
                                          "      JOIN pacco AS p ON b.id_contenuto = p.id "
                                          "      WHERE p.tipo_carta = $2 AND p.dedica = $3"
                                          "      GROUP BY b.id_bundle "
                                          ") as b ON p.id=b.id_bundle "
                                          "GROUP BY cf";
                    res = PQprepare(conn, "query4c", query4c, 3, NULL);

                    int carta;
                    printf("Scegli il tipo di carta. Le opzioni sono:\n");
                    for (int i = 0; i < 3; i++)
                    {
                        printf("\t%d. %s\n", i + 1, TIPO_CARTA[i]);
                    }
                    printf("Quale tipo vuoi? (1-3): ");
                    scanf("%d", &carta);
                    while (carta < 0 || carta > 3)
                    {
                        printf("\tHai scelto... molto MALE!\n\tTrovo insopportabile la sua mancanza di Input corretto\n");
                        printf("Quale carta vuoi? (1-3): ");
                        scanf("%d", &carta);
                    }

                    char dedica[100];
                    printf("Inserisci la dedica: ");
                    scanf("%99s", dedica);

                    const char *paramValues[3] = {
                        costo_str,
                        TIPO_CARTA[carta - 1],
                        dedica};
                    res = PQexecPrepared(conn, "query4c", 3, paramValues, NULL, NULL, 0);
                }
            }

            checkResults(res, conn);

            // Stampa i risultati
            int rows = PQntuples(res);
            int cols = PQnfields(res);
            printf("Corriere\t\t\tNumero pacchi\n");
            printf("--------\t\t\t------\n");

            // printf("\nCorrieri con bundle con pacchi con %s e costo maggiore di %d", TIPO_ASSICURAZIONE[assicurazione - 1], costo);
            printResult(rows, cols, res);
            PQclear(res);
        }

        if (scelta == 5)
        {
            const char *query5 = "SELECT t1.id_pacco, t1.status "
                                    "FROM tracking t1 "
                                    "INNER JOIN ("
                                    "    SELECT id_pacco, MAX(data_ora) as max_data_ora "
                                    "    FROM tracking "
                                    "    GROUP BY id_pacco"
                                    ") t2 ON t1.id_pacco = t2.id_pacco AND t1.data_ora = t2.max_data_ora;";
            PGresult *res = PQexec(conn, query5);
            checkResults(res, conn);

            // Stampa i risultati
            int rows = PQntuples(res);
            int cols = PQnfields(res);

            printf("\nRisultati della vista ultimo_tracking:\n");
            printf("ID Pacco\t\t\tStatus\n");
            printf("--------\t\t\t------\n");

            printResult(rows, cols, res);
            PQclear(res);
        }

        printf("\n\nVuoi continuare? (0=si, 1=no): ");
        scanf("%d", &end);
    }
    printf("Grazie per aver usato il mio programma! Che l'algebra relazionale sia con te\n");
    // Pulizia e chiusura
    PQfinish(conn);

    return 0;
}
\end{lstlisting}
\end{comment}

Il codice code.c permette di connettersi al database PostGres su Supabase, scelto perchè permette di lavorare in condivisione.\\
Una volta effettuata la connessione, il programma permette di eseguire 5 query diverse, a scelta dell'utente.\\
Tre delle 5 query sono state scritte in modo da essere parametrizzate, in modo da evitare SQL injection, questo grazie agli \textit{enum} che restringono la scelta dei parametri a solo quelli possibili.
\begin{itemize}
            \setlength{\itemindent}{+.2in}
            \item Media del costo dei pacchi di un corriere di un certo grado che sono passati per una filiale di un certo tipo
                  Parametrizzabile per il tipo di filiale e grado del corriere.
            \item Numero pacchi per bundle con un determinato tipo di assicurazione
                  Parametrizzabile per il tipo di assicurazione.
            \item Corrieri con tempi di consegna maggiori alla media.
            \item Corrieri con bundle con pacchi assicurati o regalo e costo specifici. 
                Parametrizzabile per tutti i valori \textit{Assicurati e Regalo} e per il Costo.
            \item Status attuale dei pacchi
\end{itemize}