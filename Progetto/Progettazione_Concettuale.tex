\section{Progettazione Concettuale}

Il Diagramma \ref{ER} riassume i requisiti della sezione 2.\\ Esistono 3 tipi di Pacchi: Bundle, Pacchi Assicurati, Pacchi Regalo.
Bundle risulta particolarmente singolare in quanto implementa una struttura ricorsiva. Per natura infatti un bundle è un insieme di elementi, nel caso specifico di pacchi.\\
Necessita particolari attenzioni anche l'entità `Tracking`. Questa rappresenta sostanzialmente uno storico degli aggiornamenti nel tempo delle informazioni di status e posizione di un pacco. Per ogni DataOra di un determinato Pacco viene infatti associato uno Status, una Posizione presso le filiali dell'azienda di spedizione e eventualmente anche una Nota come descritto sopra. L'entità pacco invece manterrà solo l'ultimo degli aggiornamenti di stato, ovvero quello che nell'insieme dei Tracking avrà il suo Id e il valore maggiore di DataOra.\\
La relazione Sede indica rispettivamente la sede di appartenza di un Corriere, specificando se si tratti della sede attuale o passata attraverso l'attributo booleano ATTUALE. Si tratta quindi di una relazione molti a molti che rappresenta lo storico delle sedi a cui è appartenuto un Corriere. Trattandosi di una relazione N a N non è comparsa nell'Analisi dei requisiti e verrà gestita con la creazione di una tabella SEDE.
